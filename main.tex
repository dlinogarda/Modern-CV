% FortySecondsCV LaTeX template
% Copyright © 2019 René Wirnata <rene.wirnata@pandascience.net>
% Licensed under the 3-Clause BSD License. See LICENSE file for details.
%
% Attributions
% ------------
% * fortysecondscv is based on the twentysecondcv class by Carmine Spagnuolo 
%   (cspagnuolo@unisa.it), released under the MIT license and available under
%   https://github.com/spagnuolocarmine/TwentySecondsCurriculumVitae-LaTex
% * further attributions are indicated immediately before corresponding code

%-------------------------------------------------------------------------------
%                             ADDITIONAL PACKAGES
%-------------------------------------------------------------------------------
\documentclass[
  a4paper, 
%   showframes,
   maincolor=cvblue,
   sectioncolor=cvblue,
%  subsectioncolor=orange
%   sidebarwidth=0.4\paperwidth,
%   topbottommargin=0.03\paperheight,
%   leftrightmargin=20pt
]{fortysecondscv}

% improve word spacing and hyphenation
\usepackage{microtype}
\usepackage{ragged2e}
\usepackage{rotating}

% take care of proper font encoding
\ifxetex
	\usepackage{fontspec}
	\defaultfontfeatures{Ligatures=TeX}
% \newfontfamily\headingfont[Path = fonts/]{segoeuib.ttf} % local font
\else
	\usepackage[utf8]{inputenc}
	\usepackage[T1]{fontenc}
% \usepackage[sfdefault]{noto} % use noto google font
\fi

% enable mathematical syntax for some symbols like \varnothing
\usepackage{amssymb}

% bubble diagram configuration
\usepackage{smartdiagram}
\smartdiagramset{
  % defaut font size is \large, so adjust to harmonize with sidebar layout
  bubble center node font = \footnotesize,
  bubble node font = \footnotesize,
  % default: 4cm/2.5cm; make minimum diameter relative to sidebar size
  bubble center node size = 0.3\sidebartextwidth,
  bubble node size = 0.25\sidebartextwidth,
  distance center/other bubbles = 1.5em,
  % set center bubble color
  bubble center node color = maincolor!70,
  % define the list of colors usable in the diagram
  set color list = {maincolor!10, maincolor!40,
  maincolor!20, maincolor!60, maincolor!35},
  % sets the opacity at which the bubbles are shown
  bubble fill opacity = 0.8,
}


%-------------------------------------------------------------------------------
%                            PERSONAL INFORMATION
%-------------------------------------------------------------------------------
% profile picture
\cvprofilepic{pics/closeup_lino3}
% your name
\cvname{\LARGE Denaro, Lino Garda}
% job title/career
%\cvjobtitle{Assistant Professor of\\[0.2em] Robotics Engineering}
\cvjobtitle{Postdoctoral Fellow at\\[0.2em] National Taiwan University}

% date of birth
%\cvbirthday{\textbf{Date of Birth}: Sep. X, XXXX}
% short address/location, use \newline if more than 1 line is required

% phone number
%\cvphone{XXXXXXXX}
% email address
\cvaddress{dlinogarda@ntu.edu.tw}
\cvmail{dlinogarda@tutanota.com}
% personal website
\cvsite{\href{https://dlinogarda.github.io/}{\textbf{web}: https://dlinogarda.com/}}
% pgp key
\cvkey{ORCID}{0000-0001-5367-5392} % put your actual ORCID here in the two brackets
\cvwos{\href{https://www.webofscience.com/wos/author/rid/AAM-9080-2020}{\textbf{WOS}: AAM-9080-2020}}
% https://www.webofscience.com/wos/author/rid/AAM-9080-2020
% add additional information
% \newcommand{\additional}{some more?}


%-------------------------------------------------------------------------------
%                              SIDEBAR 1st PAGE
%-------------------------------------------------------------------------------
% overwrite default icons and order of personal information
% \renewcommand{\personaltable}{%
% 	\begin{personal}[0.8em]
% 		\circleicon{\faKey}      & \cvkey  \\
% 		\circleicon{\faAt}       & \cvmail \\
% 		\circleicon{\faGlobe}    & \cvsite \\
% 		\circleicon{\faPhone}    & \cvphone \\
% 		\circleicon{\faEnvelope} & \cvaddress \\
% 		\circleicon{\faInfo}     & \cvbirthday \\
% 		% add another line
% 		\circleicon{\faQuestion} & \additional
% 	\end{personal}
% }

% add more profile sections to sidebar on first page
\addtofrontsidebar{
	% include gosquare national flags from https://github.com/gosquared/flags;
	% naming according to ISO 3166-1 alpha-2 country codes
	\graphicspath{{pics/flags/}}
	\profilesection{Interests}
		\skill{\faGraduationCap}{Remote Sensing}
		\skill{\faGraduationCap}{Atmospheric Correction}
		\skill{\faGraduationCap}{Change Detection Analysis}
		\skill{\faGraduationCap}{Artificial Intelligence}
				\skill{\faGraduationCap}{Convolution Neural Network}


	\profilesection{Overall Skills}
	\chartlabel{Programming:}\\
		\pointskill{}{MATLAB}{5}
 		\pointskill{}{Python}{5}
		\pointskill{}{C\#, CSS}{4}
    \chartlabel{Tools:}\\
    	\pointskill{}{OpenCV}{4}
		\pointskill{}{TensorFlow}{5}
		\pointskill{}{Keras}{4}
	\chartlabel{Remote Sensing Tools:}\\
 		\pointskill{}{ENVI, SNAP}{5}
	 	\pointskill{}{BEAM VISAT}{5}
	 	\pointskill{}{Arc MAP}{5}
	 	\pointskill{}{QGIS}{4}
	\chartlabel{Other Tools:}\\
		\pointskill{}{LaTeX}{5}
		\pointskill{}{AutoDesk Map}{5}
		\pointskill{}{3D Modelling}{3}
		\pointskill{}{2D Modelling}{4}
}


%-------------------------------------------------------------------------------
%                              SIDEBAR 2nd PAGE
%-------------------------------------------------------------------------------
\addtobacksidebar{
	\profilesection{Short Bio}
	\aboutme{
During Lino's studies from 2014 - 2022, He has attended many local and international conferences among countries, such as Korea, Japan, Philippine, Taiwan, and Indonesia. He also has been publishing international papers and still going on. After graduated in Doctoral degree, He continue working as Postdoctoral Fellow in National Taiwan University (NTU) Taiwan.}

	\profilesection{Metrics}
% 	\chartlabel{Bubble Diagram}
	\begin{figure}\centering
		\smartdiagram[bubble diagram]{\large 
% 			\textcolor{white}{\textbf{Being a}} \\ 
% 			\textcolor{white}{\textbf{Panda}}, % center bubble	
			\textcolor{black!90}{h-index}\\
			\textcolor{black!90}{\Large 4},
			\textcolor{black!90}{Citations}\\
			\textcolor{black!90}{50},
			\textcolor{black!90}{RG Score}\\
			\textcolor{black!90}{7.32},
			\textcolor{black!90}{Publications}\\
			\textcolor{black!90}{14}
% 			\textcolor{black!90}{Playing},
% 			\textcolor{black!90}{Chilling}
		}
	\end{figure}

\profilesection{Profiles}
\centering
\href{https://www.linkedin.com/in/lino-garda-denaro-ms-phd}{\includegraphics[width=0.2\sidebartextwidth]{in.png}}\quad \href{https://www.researchgate.net/profile/Lino-Denaro}{\includegraphics[width=0.2\sidebartextwidth]{pics/rg.png}}\quad  \href{https://github.com/dlinogarda}{\includegraphics[width=0.2\sidebartextwidth]{github.png}}
	\profilesection{Languages}
	\barskill{}{\textbf{Indonesian} (Mother Tongue)}{100}
	\barskill{}{\textbf{English} (TOEFL ITP:523, 2018)}{80}
	\barskill{}{\textbf{Chinese} (Intro. course, 2016-17)}{10}


	\profilesection{Personal}
	\begin{memberships}
 		\membership{}{Lino was born in Batam, Indonesia. He received the B.Eng. degree in Department of Geomatics Engineering from Institute of Technology Sepuluh Nopember (ITS) Surabaya, Indonesia (2011-2015). Afterward, He got a scholarship in Taiwan to continue studying in the geomatics field for his master's degree (2016-2018) and followed by his doctoral degree (2018-2022). He graduated his Doctoral degree program in February 2022 with 6 papers and 1 paper in the submission process.}
 	%	\membership{}{}
	\end{memberships}


}

% KEYWORD
\addtothirdsidebar{
\textbf{}\\
\begin{turn}{-90}\chartlabel{Remote Sensing}\chartlabel{Pseudo-invariant Features (PIFs)}\chartlabel{CCA}\chartlabel{GCCA}\chartlabel{KCCA}\chartlabel{Change Detection Analysis}\chartlabel{Normalization}\chartlabel{Optimization}\end{turn}

}

%-------------------------------------------------------------------------------
%                         TABLE ENTRIES RIGHT COLUMN
%-------------------------------------------------------------------------------
\begin{document}
\newpage
\restoregeometry
\sidebarwidth=0.35\paperwidth

\makefrontsidebar
\cvsection{Lino Garda Denaro, B.Eng., M.S., Ph.D.}
%\cvsection{\small \hfill Lino Garda Denaro, B.Eng., M.S., Ph.D.}
%\cvsection{\hfill Lino Garda Denaro, B.Eng., M.S., Ph.D.}
%\tiny \scriptsize \footnotesize \small \normalsize \large \Large \LARGE \huge \Huge

\cvsection{Professional Experience}
\begin{cvtable}
    \cvitem{Jul, 2022 -- Present}{\color{cvsectioncolor}Postdoctoral Fellow}{\href{https://www.google.com/maps/place/Taipei+City/@25.0855451,121.4914613,12z/data=!3m1!4b1!4m5!3m4!1s0x3442ac6b61dbbd8b:0xbcd1baad5c06a482!8m2!3d25.0329636!4d121.5654268}{\textbf{Taipei, Taiwan}}}{
    \href{https://www.geog.ntu.edu.tw/index.php/en/}{\textbf{Department of Geography}}\\
	\href{https://www.ntu.edu.tw/english/}{\textit{National Taiwan University (NTU)}}
    %Working for hyperspectral predictions for SWIR-2 and solar-induced chlorophyll fluorescence (SIF)
    }
    \cvitem{Feb, 2018 -- Feb, 2022}{\color{cvsectioncolor}Student Research Assistant}{\href{https://www.google.com/maps/place/Tainan,+East+District,+Tainan+City/@23.1229945,120.1309569,11z/data=!3m1!4b1!4m5!3m4!1s0x346e76935135f22b:0x1cd5fa34185ab69c!8m2!3d22.9997281!4d120.2270277}{\textbf{Tainan, Taiwan}}}{
    \textit{Geo-Artificial Intelligence Laboratory}\\
    \href{https://www.geomatics.ncku.edu.tw/english/}{\textbf{Department of Geomatics}}\\
	\href{https://www.ncku.edu.tw/}{\textit{National Cheng Kung University (NCKU)}}
    }
    \cvitem{Feb, 2016 -- Feb, 2018}{\color{cvsectioncolor}Student Research Assistant}{\href{https://www.google.com/maps/place/Tainan,+East+District,+Tainan+City/@23.1229945,120.1309569,11z/data=!3m1!4b1!4m5!3m4!1s0x346e76935135f22b:0x1cd5fa34185ab69c!8m2!3d22.9997281!4d120.2270277}{\textbf{Tainan, Taiwan}}}{
    \textit{Digital Geometry Laboratory}\\
    \href{https://www.geomatics.ncku.edu.tw/english/}{\textbf{Department of Geomatics}}\\
	\href{https://www.ncku.edu.tw/}{\textit{National Cheng Kung University (NCKU)}}
    }
    \cvitem{Sep, 2015 -- Feb, 2016}{\color{cvsectioncolor}Working in Goverment Institution}{\href{https://www.google.com/maps/place/Surabaya,+Surabaya+City,+East+Java,+Indonesia/@-7.2754438,112.6424721,12z/data=!3m1!4b1!4m5!3m4!1s0x2dd7fbf8381ac47f:0x3027a76e352be40!8m2!3d-7.2574719!4d112.7520883}{\textbf{Surabaya, Indonesia}}}{
    \textbf{Mapping and Development}\\
    \href{https://bappeko.surabaya.go.id/}{\textit{Department of Development Planning Agency of Surabaya City}}
	%\href{https://www.ncku.edu.tw/}{\textit{National Cheng Kung University (NCKU)}}
    }
    \cvitem{May, 2014 -- Sep, 2014}{\color{cvsectioncolor}Internship in Department of Remote Sensing}{\href{https://www.google.com/maps/place/Jakarta,+Indonesia/@-6.2295712,106.759478,12z/data=!3m1!4b1!4m5!3m4!1s0x2e69f3e945e34b9d:0x5371bf0fdad786a2!8m2!3d-6.2087634!4d106.845599}{\textbf{Jakarta, Indonesia}}}{
    \textbf{Remote Sensing and Technology Center}\\
    \href{https://www.lapan.go.id/}{\textit{Lembaga Penerbangan dan Antariksa Nasional (LAPAN)}}
	%\href{https://www.ncku.edu.tw/}{\textit{National Cheng Kung University (NCKU)}}
    }
\end{cvtable}

\cvsection{Education}
%\subsection{Postgraduate Studies}
\begin{cvtable}
	\cvitem{2018 -- 2022}{\color{cvsectioncolor}Doctor of Philosophy}{\href{https://www.google.com/maps/place/Tainan,+East+District,+Tainan+City/@23.1229945,120.1309569,11z/data=!3m1!4b1!4m5!3m4!1s0x346e76935135f22b:0x1cd5fa34185ab69c!8m2!3d22.9997281!4d120.2270277}{\textbf{Tainan, Taiwan}}}
		{
		\href{https://www.geomatics.ncku.edu.tw/english/}{\textbf{Department of Geomatics}}\\
		\href{https://www.ncku.edu.tw/}{\textit{National Cheng Kung University (NCKU)}}\\
		\textbf{GPA: } 3.9
		%\chartlabel{HRI} \chartlabel{Teleoperation} \chartlabel{Shared Control} \chartlabel{Eye Gaze}
		}
		
	\cvitem{2016 -- 2018}{\color{cvsectioncolor}Master of Science}{\href{https://www.google.com/maps/place/Tainan,+East+District,+Tainan+City/@23.1229945,120.1309569,11z/data=!3m1!4b1!4m5!3m4!1s0x346e76935135f22b:0x1cd5fa34185ab69c!8m2!3d22.9997281!4d120.2270277}{\textbf{Tainan, Taiwan}}}
		{
		\href{https://www.geomatics.ncku.edu.tw/english/}{\textbf{Department of Geomatics}}\\
		\href{https://www.ncku.edu.tw/}{\textit{National Cheng Kung University (NCKU)}}\\
		\textbf{GPA: } 3.9
		%\chartlabel{HRI} \chartlabel{Teleoperation} \chartlabel{Shared Control} \chartlabel{Eye Gaze}
		}
\end{cvtable}

\begin{cvtable}
    \cvitem{2011 -- 2015}{\color{cvsectioncolor}Bachelor of Engineering}{\href{https://www.google.com/maps/place/Surabaya,+Surabaya+City,+East+Java,+Indonesia/@-7.2754438,112.6424721,12z/data=!3m1!4b1!4m5!3m4!1s0x2dd7fbf8381ac47f:0x3027a76e352be40!8m2!3d-7.2574719!4d112.7520883}{\textbf{Surabaya, Indonesia}}}
		{
		\href{https://www.its.ac.id/tgeomatika/}{\textbf{Department of Geomatics Engineering}}\\
		\href{https://www.its.ac.id/}{\textit{Institute Technology of Sepuluh Nopember (ITS)}}\\
		\textbf{GPA: } 3.38
		%\chartlabel{HRI} \chartlabel{Teleoperation} \chartlabel{Shared Control} \chartlabel{Eye Gaze}
		}
    \cvitem{2008 -- 2011}{\color{cvsectioncolor}National High School}{\href{https://www.google.com/maps/place/Kediri,+Kediri+Regency,+East+Java,+Indonesia/@-7.8452475,111.9751594,12.75z/data=!4m5!3m4!1s0x2e78570dfd6e0647:0x25037b968333d9b2!8m2!3d-7.8480156!4d112.0178286}{\textbf{Kediri, Indonesia}}}
		{
		\textbf{Natural Science Class 2}\\
		\href{http://sman4kediri.siap-sekolah.com/kurikulum-akademik/\#.Y2CWXnZByHs}{\textit{SMA Negeri 4}}%\\
		%\textbf{Score: } 83
		}
\end{cvtable}

\cvsection{Skills}
{\color{cvsectioncolor}\textbf{Remote Sensing}} %\vskip0pt 
\begin{itemize}
	{\small
	\item Change Detection Analysis, Land Analysis, Ocean Color (Chl-a, Total Suspended Solid), Air Pollution (PM2.5,10), Ground Water Analysis, Hot-spot Area, Environmental Analysis
	}
\end{itemize}
{\color{cvsectioncolor}\textbf{Digital Image Processing}} 
\begin{itemize}
	{\small
	\item Histogram Processing, Image Filtering (Frequency and Spatial Domains using Fourier Transform), Morphological Image, Restoration and Reconstruction, Image Segmentation, Feature Extraction, and Image Pattern Classification
	}
\end{itemize}
{\color{cvsectioncolor}\textbf{Programming}}
\begin{itemize}
	{\small
	\item MATLAB, Python (Anaconda, PyCharm, Jupyter, Colab), C\#, SQL, R, Java, JavaScript
	}
\end{itemize}
{\color{cvsectioncolor}\textbf{Machine Learning}}
\begin{itemize}
	{\small
	\item Deep Learning (Dense Neural Network, Convolutional Neural Network), Water-Net, U-Net, Reinforcement Learning, Transfer Learning
	}
\end{itemize}
{\color{cvsectioncolor}\textbf{Computer Programs}}
\begin{itemize}
	{\small
	\item Remote Sensing Software (BEAM Visat, SNAP, ENVI, Q-GIS, Arc-Map, etc), Video Editor (Adobe premier Pro), 3D Modelling (PIX 4D Mapper, AutoCAD, Agisoft Photoscan, Australis, Cloud Compare), 2D Design (Corel Draw, Photoshop), Office Applications, LaTeX
	}
\end{itemize}

\newpage
\makebacksidebar

\cvsection{Reviewer}
\begin{itemize}
	{
	\item IEEE - Journal of Selected Topics in Applied Earth Observations and Remote Sensing
	\item IEEE - Transaction on Geoscience and Remote Sensing
	\item IEEE - International Journal of Remote Sensing
	}
\end{itemize}

\cvsection{Awards \& Achievements}
\begin{cvtable}
	\cvitem{2018 -- 2022}{Ph.D Distinguished Scholarship - NCKU, Taiwan}{}{}\\
	\cvitem{2018 -- 2022}{Ministry of Science and Technology (MOST) Scholarship, Taiwan}{}{}\\
	\cvitem{2019}{Invited Speaker of PhilGeos Conference, Philippine}{}{}\\
	\cvitem{2017}{Best Presenter of International Conferences of Indonesian Society for Remote Sensing (ICOIRS), Indonesia}{}{}\\
	\cvitem{2016 -- 2018}{M.Sc Distinguished Scholarship - NCKU, Taiwan}{}{}\\
	\cvitem{2016 -- 2018}{Ministry of Science and Technology (MOST) Scholarship, Taiwan}{}{}\\
	\cvitem{2015}{Analyst Surveyor License (Indonesian Surveyors Association)}{}{}\\
	\cvitem{2013 -- 2015}{Van Deventer-Maas Scholarship (VDMS)}{}{}
\end{cvtable}

\cvsection{Organizations}
\begin{cvtable}%\vskip10pt
	\cvitem{2019 -- 2020}{Secretary of Dewan Perwakilan Mahasiswa, PPI Taiwan}{}{}\\
	\cvitem{2018 -- 2019}{Department of Foreign Affair, PPI Tainan}{}{}\\
	\cvitem{2017 -- 2018}{Head Department Islamic Study of MSA, NCKU Taiwan}{}{}\\
	\cvitem{2016 -- 2017}{Head Department Islamic Study of MSA, NCKU Taiwan}{}{}\\
	\cvitem{2014 -- 2015}{Supervisor at Kost Quran SDM-IPTEK, Surabaya Indonesia}{}{}\\
	\cvitem{2012 -- 2014}{Department of Ristek FTSP ITS, Surabaya Indonesia}{}{}
\end{cvtable}

\newpage
\makethirdsidebar
\sidebarwidth=.08\paperwidth
\newgeometry{right=1cm,left=2cm,bottom=0.1cm,top=1cm}

\cvsection{Publications}
\begin{cvtable}
    \cvitem{Feb, 2020}{\color{cvsectioncolor}\href{https://ieeexplore.ieee.org/xpl/RecentIssue.jsp?punumber=4609443}{IEEE Journal of Selected Topics in Applied Earth Observations and Remote Sensing}}{}{
	\href{https://ieeexplore.ieee.org/document/9016141}{\textit{Hybrid canonical correlation analysis and regression for radiometric normalization of cross-sensor satellite imagery}}
    }
    \cvitem{Jul, 2019}{\color{cvsectioncolor}\href{https://igarss2019.org/}{IGARSS 2019- IEEE International Geoscience and Remote Sensing Symposium}}{}{
	\href{https://a-a-r-s.org/proceeding/ACRS2019/WeC4-1.pdf}{\textit{Nonlinear relative radiometric normalization for Landsat 7 and Landsat 8 imagery}}
    }
    \cvitem{Mar, 2019}{\color{cvsectioncolor}\href{https://ieeexplore.ieee.org/xpl/RecentIssue.jsp?punumber=8859}{IEEE Geoscience and Remote Sensing Letters}}{}{
	\href{https://ieeexplore.ieee.org/document/8657354}{\textit{Pseudoinvariant feature selection using multitemporal MAD for optical satellite images}}
    }
    \cvitem{Feb, 2019}{\color{cvsectioncolor}\href{https://www.isprs.org/publications/archives.aspx}{The International Archives of Photogrammetry, Remote Sensing and Spatial Information Sciences}}{}{
	\href{http://ieeexplore.ieee.org/document/9016141}{\textit{Hybrid Canonical Correlation Analysis and Regression for Radiometric Normalization of Cross-Sensor Satellite Images}}
    }
    \cvitem{Nov, 2019}{\color{cvsectioncolor}\href{https://www.sciencedirect.com/}{ISPRS Journal of Photogrammetry and Remote Sensing}}{}{
	\href{https://www.sciencedirect.com/science/article/abs/pii/S0924271618303034}{\textit{Spectral-consistent relative radiometric normalization for multitemporal Landsat 8 imagery}}
    }
    \cvitem{Aug, 2018}{\color{cvsectioncolor}\href{https://www.spiedigitallibrary.org/}{Journal of Applied Remote Sensing}}{}{
	\href{https://www.spiedigitallibrary.org/journals/journal-of-applied-remote-sensing/volume-12/issue-4/045002/Pseudoinvariant-feature-selection-for-cross-sensor-optical-satellite-images/10.1117/1.JRS.12.045002.full?SSO=1}{\textit{Pseudoinvariant feature selection for cross-sensor optical satellite images}}
    }
\end{cvtable}

\cvsection{Conferences}
\begin{cvtable}
	\cvitem{$21-24^{th}$ Nov 2021}{Asian Conference on Remote Sensing (ACRS), Vietnam}{}{}\\
	\cvitem{$14-15^{th}$ Nov 2019}{PhilGeos (GeoAdvences) Paper 1, Philippine}{}{}\\
	\cvitem{$16-17^{th}$ Nov 2019}{PhilGeos (GeoAdvences) Paper 2, Philippine}{}{}\\
	\cvitem{$14-18^{th}$ Oct 2019}{Asian Conference on Remote Sensing (ACRS), Korea}{}{}\\
	\cvitem{$17-19^{th}$ Apr 2019}{International Symposium on Remote Sensing (ISRS), Taiwan}{}{}\\
	\cvitem{$22-27^{th}$ Jul 2018}{IEEE International Geoscience and Remote Sensing Symposium (IGARSS), Japan}{}{}\\
	\cvitem{$07-10^{th}$ Aug 2018}{International conference of SG37, Taiwan}{}{}\\
	\cvitem{$31-01^{th}$ Oct 2017}{International Conference of Indonesia Society for Remote Sensing (ICOIRS), Indonesia}{}{}\\
	\cvitem{$05-07^{th}$ Des 2016}{International Symposium on GNSS (ISGNSS), Taiwan}{}{}\\
	\cvitem{$12-15^{th}$ Jul 2016}{International conference of SG36, Taiwan}{}{}
\end{cvtable}
\cvsignature
\end{document} 
